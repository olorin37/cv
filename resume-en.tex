%%%%%%%%%%%%%%%%%
% This is an sample CV template created using altacv.cls
% (v1.7, 9 August 2023) written by LianTze Lim (liantze@gmail.com). Compiles with pdfLaTeX, XeLaTeX and LuaLaTeX.
%
%% It may be distributed and/or modified under the
%% conditions of the LaTeX Project Public License, either version 1.3
%% of this license or (at your option) any later version.
%% The latest version of this license is in
%%    http://www.latex-project.org/lppl.txt
%% and version 1.3 or later is part of all distributions of LaTeX
%% version 2003/12/01 or later.
%%%%%%%%%%%%%%%%

%% Use the "normalphoto" option if you want a normal photo instead of cropped to a circle
% \documentclass[10pt,a4paper,normalphoto]{altacv}

\documentclass[10pt,a4paper,ragged2e,withhyper]{altacv}
%% AltaCV uses the fontawesome5 and packages.
%% See http://texdoc.net/pkg/fontawesome5 for full list of symbols.

% Change the page layout if you need to
\geometry{left=1.25cm,right=1.25cm,top=1.5cm,bottom=1.5cm,columnsep=1.2cm}

% The paracol package lets you typeset columns of text in parallel
\usepackage{paracol}

% Change the font if you want to, depending on whether
% you're using pdflatex or xelatex/lualatex
% WHEN COMPILING WITH XELATEX PLEASE USE
% xelatex -shell-escape -output-driver="xdvipdfmx -z 0" sample.tex
\ifxetexorluatex
  % If using xelatex or lualatex:
  \setmainfont{Roboto Slab}
  \setsansfont{Lato}
  \renewcommand{\familydefault}{\sfdefault}
\else
  % If using pdflatex:
  \usepackage[rm]{roboto}
  \usepackage[defaultsans]{lato}
  % \usepackage{sourcesanspro}
  \renewcommand{\familydefault}{\sfdefault}
\fi

% Change the colours if you want to
\definecolor{SlateGrey}{HTML}{2E2E2E}
\definecolor{LightGrey}{HTML}{666666}
\definecolor{DarkPastelRed}{HTML}{450808}
\definecolor{PastelRed}{HTML}{8F0D0D}
\definecolor{GoldenEarth}{HTML}{E7D192}
\colorlet{name}{black}
\colorlet{tagline}{PastelRed}
\colorlet{heading}{DarkPastelRed}
\colorlet{headingrule}{GoldenEarth}
\colorlet{subheading}{PastelRed}
\colorlet{accent}{PastelRed}
\colorlet{emphasis}{SlateGrey}
\colorlet{body}{LightGrey}

% Change some fonts, if necessary
\renewcommand{\namefont}{\Huge\rmfamily\bfseries}
\renewcommand{\personalinfofont}{\footnotesize}
\renewcommand{\cvsectionfont}{\LARGE\rmfamily\bfseries}
\renewcommand{\cvsubsectionfont}{\large\bfseries}


% Change the bullets for itemize and rating marker
% for \cvskill if you want to
\renewcommand{\cvItemMarker}{{\small\textbullet}}
\renewcommand{\cvRatingMarker}{\faCircle}
% ...and the markers for the date/location for \cvevent
% \renewcommand{\cvDateMarker}{\faCalendar*[regular]}
% \renewcommand{\cvLocationMarker}{\faMapMarker*}


% If your CV/résumé is in a language other than English,
% then you probably want to change these so that when you
% copy-paste from the PDF or run pdftotext, the location
% and date marker icons for \cvevent will paste as correct
% translations. For example Spanish:
% \renewcommand{\locationname}{Ubicación}
% \renewcommand{\datename}{Fecha}


%% Use (and optionally edit if necessary) this .tex if you
%% want to use an author-year reference style like APA(6)
%% for your publication list
% % When using APA6 if you need more author names to be listed
% because you're e.g. the 12th author, add apamaxprtauth=12
\usepackage[backend=biber,style=apa6,sorting=ydnt]{biblatex}
\defbibheading{pubtype}{\cvsubsection{#1}}
\renewcommand{\bibsetup}{\vspace*{-\baselineskip}}
\AtEveryBibitem{%
  \makebox[\bibhang][l]{\itemmarker}%
  \iffieldundef{doi}{}{\clearfield{url}}%
}
\setlength{\bibitemsep}{0.25\baselineskip}
\setlength{\bibhang}{1.25em}


%% Use (and optionally edit if necessary) this .tex if you
%% want an originally numerical reference style like IEEE
%% for your publication list
\usepackage[backend=biber,style=ieee,sorting=ydnt,defernumbers=true]{biblatex}
%% For removing numbering entirely when using a numeric style
\setlength{\bibhang}{1.25em}
\DeclareFieldFormat{labelnumberwidth}{\makebox[\bibhang][l]{\itemmarker}}
\setlength{\biblabelsep}{0pt}
\defbibheading{pubtype}{\cvsubsection{#1}}
\renewcommand{\bibsetup}{\vspace*{-\baselineskip}}
\AtEveryBibitem{%
  \iffieldundef{doi}{}{\clearfield{url}}%
}


%% sample.bib contains your publications
\addbibresource{sample.bib}

%%%
\newcommand{\CompanyName}{\hspace{4.5em}}
%%%

\begin{document}

\name{Jakub A. Gramsz}
\tagline{DevOps Architect / Software Engineer}
%% You can add multiple photos on the left or right
% \photoR{2.8cm}{Globe_High}
% \photoL{2.5cm}{Yacht_High,Suitcase_High}
\photoR{2.8cm}{JakubGramsz_sq}

\personalinfo{%
  % Not all of these are required!
  \email{jakub@gramsz.com}
  \phone{+48 691 777 885}
  \mailaddress{ul. Kokosowa 3/56, 54-060 Wrocław}
  \location{Polska, POLAND}
  \homepage{gramsz.com}
  \linkedin{jakub-gramsz-376223101}
  \github{olorin37}
  %\orcid{0000-0000-0000-0000}
  %% You can add your own arbitrary detail with
  %% \printinfo{symbol}{detail}[optional hyperlink prefix]
  % \printinfo{\faPaw}{Hey ho!}[https://example.com/]

  %% Or you can declare your own field with
  %% \NewInfoFiled{fieldname}{symbol}[optional hyperlink prefix] and use it:
  % \NewInfoField{gitlab}{\faGitlab}[https://gitlab.com/]
  % \gitlab{your_id}
  %%
  %% For services and platforms like Mastodon where there isn't a
  %% straightforward relation between the user ID/nickname and the hyperlink,
  %% you can use \printinfo directly e.g.
  % \printinfo{\faMastodon}{@username@instace}[https://instance.url/@username]
  %% But if you absolutely want to create new dedicated info fields for
  %% such platforms, then use \NewInfoField* with a star:
  % \NewInfoField*{mastodon}{\faMastodon}
  %% then you can use \mastodon, with TWO arguments where the 2nd argument is
  %% the full hyperlink.
  % \mastodon{@username@instance}{https://instance.url/@username}
}

\makecvheader
%% Depending on your tastes, you may want to make fonts of itemize environments slightly smaller
% \AtBeginEnvironment{itemize}{\small}

%% Set the left/right column width ratio to 6:4.
\columnratio{0.5}

% Start a 2-column paracol. Both the left and right columns will automatically
% break across pages if things get too long.
\begin{paracol}{2}
\cvsection{Experience}

\cvevent{DevOps Architect / Technical Leader}{Nokia}{July 2017 -- current}{Wrocław}
\begin{itemize}

\item Design and development of the fully automated hardware tests for
BTS Operations \& Maintenance embedded application. Scope of the automation:
executing tests for each release candidate, preparing software package for testing,
preparing hardware for software testing, infrastructure orchestrating and provisioning,
configuration management. Technology: \cvtag{Bash,} \cvtag{Python,} \cvtag{Puppet,}
\cvtag{Jenkins} behind \cvtag{nginx,} \cvtag{Groovy,} \cvtag{Gradle,} \cvtag{Hashicorp Vault,}
\cvtag{GitLab,} \cvtag{Gerrit,} \cvtag{Terraform} with \cvtag{OpenStack,} \cvtag{Docker,}
\cvtag{Podman,} \cvtag{jq,} \cvtag{GNU/Linux,} \cvtag{S3,} \cvtag{InfluxDB,} \cvtag{Elasticsearch,}
\cvtag{Robot Framework.}

\item Process aiding web application for testers has been introduced at some
point. My responsibility was leading its development. Technology:
\cvtag{Django,} \cvtag{JavaScript,} \cvtag{PostgreSQL} (self-deployed and with AWS RDS like in-house solution).

\item Deployment and maintenance of the web application (written in \cvtag{Rust)}
for helping in memory and CPU profiling of the embedded applicaton.

\item The post involved hosting planning and technical meetings, planning work for
team members, constant code review, tasks refinement, task
prioritization for small team (of avg 4 members).

\end{itemize}

%\smalldivider
\medskip

\cvevent{DevOps Engineer}{Nokia}{Nov. 2015 -- June 2017}{Wrocław}
\begin{itemize}
\item 
  Maintenance and development of the shell scripts allowing release testing
  on hardware of the embedded application. Technology: \cvtag{Bash,} \cvtag{Nodejs,}
  \cvtag{GNU/Linux,} \cvtag{Git,} \cvtag{Gitlab,} \cvtag{Jenkins,} \cvtag{Robot Framework,} \cvtag{Eucalyptus}
  (AWS-compatible open source cloud), \cvtag{Ansible.}
\item
  The post required strong self-reliance.
\end{itemize}

%\smalldivider
\medskip

\cvevent{Integration Tools Developer}{Nokia}{Nov. 2014 -- Nov. 2015}{Wrocław}
\begin{itemize}
\item 
  Implementing elements of the integration testing environment for embedded
  application (the most complex one on BTS). Technology: \cvtag{Python,} \cvtag{Bash,} \cvtag{LXC,}
  \cvtag{VirtalBox,} \cvtag{Debian,} \cvtag{Subversion,} \cvtag{JIRA}, \cvtag{TeamCity,} \cvtag{Eclipse.}
\end{itemize}

\medskip

{\tiny I hereby authorise \CompanyName to process my personal data included in my job application for the needs of the recruitment process. I hereby authorise \CompanyName to process my personal data for the purposes of any future recruitment processes.}

% use ONLY \newpage if you want to force a page break for
% ONLY the current column
\newpage

%% Switch to the right column. This will now automatically move to the second
%% page if the content is too long.
\switchcolumn

\cvsection{Philosophy}

\begin{quote}
``Less is more.''
``Simple is better than Easy''
\end{quote}

\cvsection{Languages}

\cvskill{English}{4}
\divider

\cvskill{Polish}{5} %% Supports X.5 values.

%% Yeah I didn't spend too much time making all the
%% spacing consistent... sorry. Use \smallskip, \medskip,
%% \bigskip, \vspace etc to make adjustments.
\medskip

\cvsection{Education}

\cvevent{M.Sc.\ in Computer Science}{Wrocław University of Technology}{Oct. 2012 -- June 2014}{Wrocław}
  My specialization was \textit{Intelligent Information Systems.} Most courses was
  related to the area, but I was also interested in programming languages
  theory. I used \cvtag{Java,} \cvtag{C\#,} \cvtag{R} and \cvtag{MATLAB} for AI related projects and \cvtag{Haskell}
  for the latter. I learned basics of lambda calculus. Basic usage \cvtag{git} for
  some projects. Thesis title: “Monads in Programming Languages”,
  typesetting made with \cvtag{LaTeX.}

\divider

\cvevent{B.Sc.\ in Astronomy}{Wrocław University}{Oct. 2010 -- June 2014}{Wrocław}
  \textit{Dropped} \,
  That was my try to meet my interest in cosmology and physics… Surprisingly,
  I also had learn a lot about GNU/Linux shell (\cvtag{Bash,} \cvtag{AWK,} \cvtag{sed,} \cvtag{gnuplot)}
  during processing CCD photometer measurements made through telescope.

\divider

\cvevent{B.Sc.\ in Computer Science}{Wrocław University of Technology}{Oct. 2008 -- June 2012}{Wrocław}
  I had learned \cvtag{Java} (the main language for most projects), \cvtag{OCaml} (functional
  paradigm features), \cvtag{Haskell,} \cvtag{MATLAB,} \cvtag{Python,} \cvtag{PL/SQL} (Oracle), \cvtag{C} (with \cvtag{MPI}
  framework), \cvtag{MIPS Assembler} (the wrong choice of my mentors, as we know
  now). I had attended to extended curses stream (it was a big dose of
  mathematical and theoretical knowledge). Learning and using \cvtag{LaTeX} for
  reports and presentations. Basic usage of \cvtag{Subversion.}

\divider

\cvevent{School}{}{before July 2008}{Jelenia Góra}
  I made several small projects mixing together \cvtag{Delphi,} \cvtag{HTML,} \cvtag{PHP,} \cvtag{JavaScript,}
  \cvtag{SQL} (MySQL).

\end{paracol}


\end{document}
